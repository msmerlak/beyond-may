\emph{Model.---}Consider the dynamical system in $N$ variables
\begin{equation}\label{dynamics}
    \dot{x}_i = f(x_i) - \sum_{j}A_{ij}g(x_i)h(x_j) \, .
\end{equation}
Here $f(x_i)$ represents the self-dynamics of a population $i$ (growth and self-regulation), while $g(x_i)$ and $h(x_j)$ capture the cross-interaction of $i$ with other populations. That interaction is weighted by a coefficient $A_{ij}$, such that $A_{ij} > 0$ (resp.
$A_{ji} < 0$) implies a negative (resp.
positive) effect of $j$ on the growth of $i$. Non-zero diagonal elements $A_{ii}$ accounts for additional self-interactions with the same functional form as cross-interactions. 
We refer to $f$ as the ``production function'' and $g$ as the ``response function''.

This general setting has been used to study universality in network dynamics \cite{Barzel2013} and to construct minimal models of complex dynamics \cite{Barzel2015}, with applications ranging from biological~\cite{Alon2006,Karlebach2008} 
to social~\cite{Pastor-Satorras2001,Hufnagel2004,Dodds2005} systems.
The classic generalized Lotka-Volterra (GLV) competition model studied recently by Bunin and collaborators \cite{bunin2017ecological, biroli2018marginally} corresponds to $f$, $g$, $h$ all linear. 

However, it is natural both physically and biologically to consider more general functions, including power laws $f(x)\sim x^\alpha$, $g(x)\sim x^\beta$, $h(x) \sim x^\gamma$.
From a physical perspective, we can imagine populations $x_i$ forming three-dimensional clusters whose growth is limited to their two-dimensional surface, leading to a production function $f(x) \sim x^{2/3}$.
Biologically, the growth of organisms (populations of cells) has long been known to scale like $f(x) \sim x^k$ with $k\simeq 3/4$~\cite{Brown2004}, which can be understood in terms of hydrodynamic constraints on vascular and pulmonary networks.
For reasons that are not currently understood, a similar pattern of growth appears to recur at the level of ecological communities~\cite{Hatton2015,Hatton2023}.
In a different direction, it has been recently suggested that predator-prey interactions can be modelled with a square root law, i.e. $g(x) \sim h(x) \sim x^{1/2}$~\cite{Barbier2021,Mazzarisi2023}.