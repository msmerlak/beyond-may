Consider now Eq.~\eqref{dynamics} with i.i.d. random interactions $a_{ij}$ drawn from a distribution with mean $\mu$ and standard deviation $\sigma$; we assume the diagonal elements $a_{ii}$ have a mean value $\mu_s$ (possibly different from $\mu$) but the same standard deviation $\sigma$.
 
We compute the Jacobian matrix at equilibrium
\begin{align}
    J_{ij}^* & = a_{ij}g(x_i^*)h'(x_j^*) \qquad \qquad \textrm{for} \ i\neq j \label{eq: jac off-diag}\\
    J_{ii}^* & = f'(x_i^*) - \frac{g'(x_i^*)f(x_i^*)}{g(x_i^*)} - a_{ii}g(x_i^*)h'(x_j^*) \ , \label{eq: jac diag}
\end{align}
where we used $\sum_{j}a_{ij}g(x_i^*)h(x_j^*)=-f(x_i^*)/g(x_i^*)$.
In order to investigate the spectral properties of $J^*$, 
we follow Stone~\cite{Stone2018} and use a result in random matrix theory~\cite{Ahmadian2015} which generalize the classical `circular law'~\cite{Potters2020}.

In Ref.~\cite{Ahmadian2015} Ahmadian \emph{et al.} consider matrices of the form $M + LSR$, where $M$,  
$L$ and $R$ are deterministic matrices, and $S$ is a random matrix with i.i.d. coefficients, zero mean and variance $\sigma^2$.
They show that eigenvalues of large matrices of this form are contained in the complex domain with equation
$\textrm{Tr}[(M_\zeta M_\zeta^\dagger)^{-1}]\geq \sigma^{-2}$, where $M_\zeta = L^{-1}(\zeta I - M)R^{-1}$ and $\zeta\in\mathbb{C}$.
If $L$, $R$ and $M$ are all diagonal $N\times N$ matrices (identified with the vector of their diagonal elements), this condition simplifies to
\begin{equation}
    \sum_{i=1}^N\frac{(L_{i}R_{i})^2}{ \vert \zeta - M_{i}\vert^2 }\geq \frac{1}{\sigma^{2}} \ .
\label{eq: domain}
\end{equation}

To use this result, we decompose the interaction matrix as $a = \mu \mathbf{1} + (\mu_s-\mu)I + S$,
with $\mathbf{1}$ the matrix with all entries equal to $1$ and $S$ a random matrix as defined above.
Using rank-one perturbation theory, we can show that the $\mathbf{1}$ term does not impact stability properties and can be neglected \cite{Stone2018}.
The Jacobian in Eqs.~\eqref{eq: jac off-diag} and \eqref{eq: jac diag} thus takes the form $M + LJR$ with diagonal matrices
\begin{equation}
    M_i = f(x_i^*)\psi(x_i^*),\quad L = g(x_i^*), \quad R = h'(x_i^*).
\end{equation}
\red{Assuming all $M_i < 0$}, the domain \eqref{eq: domain} containing the eigenvalues of $J^*$ first touches the right half-plane at $\zeta = 0$, 
hence linear stability requires
\begin{equation}
    \sum_{i=1}^N \left(\frac{g(x_i^*)h'(x_i^*)}{f(x_i^*)\psi(x_i^*)}\right)^2
    < \frac{1}{\sigma^{2}}
    \label{eq: random-stability}
\end{equation}
This expression generalizes \eqref{homogeneous-general} to random interactions. 

For GLV models, i.e. $f(x_i)=x_i$, $g(x_i)=x_i$ and $h(x_i)=x_i$, \eqref{eq: random-stability} is independent of the equilibria $x^*$, and we recover the linear stability condition
$\sigma\sqrt{N} < (\mu_s-\mu)$.
More generally, any model satisfying
\begin{equation}\label{eq: equiv class glv}
    \left(\cfrac{f'(x_i^*)}{f(x_i^*)} -
        \cfrac{g'(x_i^*)}{g(x_i^*)}\right)\Bigl /
        \cfrac{g(x_i^*)h'(x_i^*)}{f(x_i^*)} = c \ ,
\end{equation}
for some constant $c$ has the same simple linear stability condition as GLV models, namely $\sigma\sqrt{N} < (\mu_s-\mu) +c$.

When the dependence on $\mathbf{x}_*$ does not cancel out, it is not possible to obtain an explicit complexity-stability relationship. 
However, introducing the distribution of equilibrium $P(x^*)$ and replacing sums with integrals in the large $N$ limit, we can write the stability condition as
\begin{equation}\label{eq: random-stability int}
    N\int dx^* frac{g(x^*)h'(x^*)}{f(x^*)\psi(x^*)} < \sigma^{-2} \ .
\end{equation}
We now discuss how $P(x_*)$ can be estimated from the statistics of interactions, focusing on the power-law case.
