\emph{Introduction.---}Few mathematical arguments have influenced biological thinking like May's prediction that large, complex ecosystems cannot be stable~\cite{May1972}.
It is, on the face of it, a perplexing conclusion.
On the one hand, May's mathematical argument is simple and seemingly model-free, suggesting universal applicability, also beyond biology \cite{Haldane2011, Moran2019}.
On the other hand, it is clear that at least some large, complex systems are stable---else which regularities would biology be studying in the first place? 
In fact, empirical observation suggests the opposite relationship between complexity and stability: species-rich, strongly-coupled communities such as rainforests tend to be stable over time, while sparser ones, for instance arctic communities, often exhibit large fluctuations, extinctions, and invasions~\cite{Hutchinson1959,Odum1959,MacArthur1955}. 
The tension between May's theoretical argument and observation is at the center of the longstanding "diversity-stability debate" in ecology \cite{McCann2000, Loreau2022}.

May's argument can be summarized as follows.
Consider a system with $N$ populations $x_i$ characterized by an equilibrium point $\mathbf x^*$.
Near that equilibrium $\mathbf x = \mathbf x^* + \delta \mathbf x$, the dynamics of the system is described by linear equations $d(\delta \mathbf x)/dt = A (\delta \mathbf x)$, and the stability of these equations requires that all eigenvalues of $A$ have negative real part.
But if $A$ can be represented as $A = B - I$, where $B$ consists of random, independent interactions (with zero 
mean and variance $\sigma^2$), and $-I$ corresponds to stabilizing self-interactions on some natural timescale, 
then the circular law of random matrix theory implies that all eigenvalues of $A$ will have negative real part only if $\sigma^2 N < 1$.
This places a sharp constraint on both diversity $N$ and interaction strength $\sigma$, often referred to as "complexity begets instability".
(This argument generalizes to $\langle B_{ij}\rangle \neq 0$, incomplete connectivity, or correlated interactions, see e.g. \cite{allesina2015stability}.)

In ecology, many authors have sought to ease the tension between May's prediction and empirical observation by invoking effects not captured by dynamical systems with random coefficients~\cite{McCann2000,Chesson2000,Mougi2012,Rohr2014,Barabas2017,Grilli2017}. 
In this letter, we use recent results in the physics of disordered systems \cite{Ahmadian2015, Roy2019} to show that May's argument itself is incomplete: in random dynamical systems, stability does not necessarily decrease with dimensionality and interaction strength---the opposite behavior is also possible, without the need for special or additional structure.
For an in-depth discussion of ecological implications of our result, see \cite{Hatton2023}.