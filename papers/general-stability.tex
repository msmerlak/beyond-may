\documentclass[%
 reprint,
%superscriptaddress,
%groupedaddress,
%unsortedaddress,
%runinaddress,
%frontmatterverbose, 
%preprint,
%preprintnumbers,
%nofootinbib,
%nobibnotes,
%bibnotes,
 amsmath,amssymb,
 aps,
%pra,
%prb,
%rmp,
%prstab,
%prstper,
%floatfix,
]{revtex4-2}

\usepackage{colortbl}
\newcommand{\red}{\textcolor{red}}

\begin{document}

\title{Complexity-stability relationships in large, complex systems}

\author{Onofrio Mazzarisi}
\author{Matteo Smerlak}

 \affiliation{MPI MiS}


\date{\today}% It is always \today, today,
             %  but any date may be explicitly specified

\begin{abstract}
Fifty years ago, Robert May predicted that large, complex systems are unlikely to be stable. Here, we revisit May's argument to show that there are, in fact, two kinds of complexity-stability relationships in disordered dynamical systems: in some cases, strong interaction and large diversity are, indeed, destabilizing; in other cases, they are stabilizing. Which class a given system falls into depends on the relative convexity of the response to self- vs. cross-interactions. We illustrate the transition between "May" (complexity begets instability) and "anti-May" (complexity begets stability) behavior with a non-linear  generalization of the Lotka-Volterra model. 
\end{abstract}


\maketitle

\section{Introduction}

Few mathematical arguments have influenced biological thinking like May's 1970 prediction that large, complex systems cannot be stable. It is, indeed, a perplexing prediction. On the one hand, the argument is so simple and compelling that it is hard not to conclude that it must hold universally. On the other, it is clear that at least some large, complex systems are stable---or else which stable patterns would biology be studying in the first place? If anything, the opposite relationship between complexity and stability seems to hold empirically: rich, strongly coupled ecosystems (eg. rainforests) tend to be  stable over time, while sparser ones (eg. arctic environments) often exhibit large fluctuations, exctinctions, and invasions. 

May's argument can be paraphrased as follows. If a system with $N$ populations $x_i$ is stable, then it can be characterized by an equilibrium $x^*$. Near that equilibrium $x = x^* + y$, the dynamics of the system will be described by linear equations $\delta\dot{y} = A y$, and stability of these equations requires that all eigenvalues of $A$ have negative real part. But if $A$ can be represented as $A = B - I$, where $B$ consists of random, independent interactions with zero mean and variance $\sigma^2$, and $-I$ corresponds to stabilizing self-interactions on some natural timescale, then the circular law of random matrix theory implies that all eigenvalues of $A$ will have negative real part only if $\sigma^2 N < 1$. This places a sharp constraint on both diversity $N$ and interaction strengh $\sigma$, often referred to as "complexity begets instability". (This argument generalizes to $\langle B_{ij}\rangle \neq 0$, to incomplete connectivity, or to correlated interactions.)

In ecology, many authors have sought to ease the tension between May's prediction and empirical observation by invoking effects not captured by dynamical systems with random coefficients: spatial or food-web structure, time delays, etc. Here, we show that May's argument is incomplete: random dynamical systems do not necessarily imply that stability decreases with diversity or interaction strength---the opposite behavior is also possible, without the need for special or additional structure. 

\section{Results}

\subsection{Model}

Consider the dynamical system in $N$ variables
\begin{equation}\label{dynamics}
    \dot{x}_i = f(x_i) + \sum_{j}a_{ij}g(x_i)h(x_j) \, .
\end{equation}
Here $f(x_i)$ represents the contribution of a population $i$ to its own growth (its "production function"), while $g(x_i)$ and $h(x_j)$ capture the interaction of $i$ with other populations. That interaction is weighted by a coefficient $a_{ij}$, such that $a_{ij} > 0$ (resp. $a_{ji} < 0$) implies a positive (resp. negative) effect of $j$ on the growth of $i$. In the following we call $g(x_i)$ the "response function". 

The classic generalized Lotka-Volterra competition model corresponds to $f$, $g$, $h$ all linear. However, it is natural both physically and biologically to consider more general functions, including power laws $f(x)\sim x^\alpha$, $g(x)\sim x^\beta$, $h(x) \sim x^\gamma$. From a physical perspective, we can imagine populations $x_i$ forming three-dimensional clusters whose growth is limited to their two-dimensional surface, leading to a production function $f(x) \sim x^{2/3}$. Biologically, the growth of organisms (populations of cells) has long been known to scale like $f(x) \sim x^k$ with $k\simeq 3/4$, which can be understood in terms of hydrodynamic constraints on vascular and pulmonary networks. For reasons that are not currently understood, a similar pattern of growth appears to recur at the community level. It is also common to model predator-prey interactions with a square root laws ($g(x) \sim h(x) \sim x^{1/2}$), for example. 

\subsection{Homogeneous interactions}

Under what condition is \eqref{dynamics} linearly stable? We begin by considering the simple case where all self-interactions have the same strength $a_{ii} = a_{\textrm{s}}$, and similarly for cross-interactions $a_{ij} = a_{\textrm{c}}$ ($i\neq j$). In that case, a straightforward computation shows that an equilibrium $x_*$ will be linearly stable if  
\begin{equation}\label{homogeneous-general}
    \left(\frac{f'_*}{f_*} - \frac{g'_*}{g_*}\right) + (a_{\textrm{s}} - a_{\textrm{c}})\,\frac{g_*h'_*}{f_*} < 0 \, . 
\end{equation}
Thus, the stability of a competitive equilibrium depends on the relative strength of self- and cross-interactions ($a_{\textrm{s}} - a_{\textrm{c}}$), but also on the relative convexity of the production and response functions ($f'_*/f_* - g'_*/g_*$). With power laws, \eqref{homogeneous-general} evaluates to $(\alpha - \beta)(N-1) < \gamma(a_{\textrm{s}}/a_{\textrm{c}}- 1) - (\alpha - \beta)(a_{\textrm{s}}/a_{\textrm{c}})$, leading to three different regimes:
\begin{itemize}
    \item If $\alpha = \beta$, stability requires $a_{\textrm{s}} > a_{\textrm{c}}$, i.e. self-interactions must be stronger than cross-interactions. This is the usual conclusion drawn from the Lotka-Volterra model. 
    \item If $\alpha > \beta$, stability places an upper bound on $N$: the more complex the system, the less likely to be stable. We can call this "May" behavior.
    \item If $\alpha < \beta$, stability places an lower bound on $N$: the more complex the system, the more likely to be stable. This is "anti-May" behavior.
\end{itemize}

\subsection{Random interactions}

Consider now Eq.~\eqref{dynamics} with i.i.d. random 
interactions $a_{ij}$ with mean $\mu$ and standard 
deviation $\sigma$, with the diagonal elements $a_{ii}$ 
extracted from a distribution with mean $\mu_s$ 
possibly different from $\mu$ and same standard deviation
as for the off-diagonal elements.  
We can compute the Jacobian matrix at equilibrium
\begin{align}
    J_{ij}^* & = a_{ij}g_*(x_i)h_*'(x_j) \qquad \qquad \textrm{for} \ i\neq j \label{eq: jac off-diag}\\
    J_{ii}^* & = f_*'(x_i) - \frac{g_*'(x_i)f_*(x_i)}{g_*(x_i)} - a_{ii}g_*(x_i)h_*'(x_j) \ , \label{eq: jac diag}
\end{align}
where we used $\sum_{j}a_{ij}g_*(x_i)h_*(x_j)=-f_*(x_i)/g_*(x_i)$.
In order to investigates its spectral properties, 
we follow Stone~[Stone2018] and use a
result in random matrix theory~[Ahmadian2015]
which generalize the classical `circular law'~[Circular].

In Ref.~[Ahmadian2015] Ahmadian \emph{et al.} 
consider matrices of the form $M + LSR$, where $M$,  
$L$ and $R$ are deterministic matrices, and $S$ 
is a random matrix with i.i.d. coefficients, 
zero mean and variance $\sigma^2$. They 
show that in the complex plane the boundary of the 
eigenspectrum of large matrices of this form is defined by
\begin{equation}
    \textrm{Tr}[(M_\zeta M_\zeta^\dagger)^{-1}]\geq \sigma^{-2} \, , 
\end{equation}
where $M_\zeta = L^{-1}(\zeta I - M)R^{-1}$ and $\zeta\in\mathbb{C}$. 
If $L$, $R$ and $M$ are all diagonal $N\times N$ matrices, 
this condition simplifies to
\begin{equation}
    \sum_{i=1}^N\frac{(L_{i}R_{i})^2}{ \vert \zeta - M_{i}\vert^2 }\geq \sigma^{-2} \ .
\label{eq: domain}
\end{equation}

Now we decompose the interaction matrix as
$a = \mu \mathbf{1} + (\mu_s-\mu)I + S$,
with $\mathbf{1}$ the rank-one matrix with all entries equal to $1$
and $S$ a random matrix as defined above.
We can limit ourselves to consider 
$a = (\mu_s-\mu)I + S$ by relying on 
rank-one perturbation theory, we would only neglect outliers
uninfluential for the linear stbaility properties of the system by doing so
(see Ref.~[Stone2018]).
The Jacobian in Eqs.~\eqref{eq: jac off-diag} and \eqref{eq: jac diag} assumes therefore the form $M + LJR$ with
\begin{align}
    M & = \textrm{diag}\left(f_*'(\mathbf x) -
    \frac{g_*'(\mathbf x)f_*(\mathbf x)}{g_*(\mathbf x)}
    +(\mu_s-\mu)g_*(\mathbf x)h_*'(\mathbf x)\right) \ , \\
    L &= \textrm{diag}(g_*(\mathbf x)) \ , \\
    R &= \textrm{diag}(h_*'(\mathbf x)) \ .
\end{align}
The domain of the eigenvalues of $J^*$, defined by
Eq.~\eqref{eq: domain}, 
first touches the right half-plane at $\zeta = 0$, 
hence linear stability requires   
\begin{equation}\label{eq: random-stability}
    \sum_{i=1}^N \cfrac{\left(\cfrac{g_*(x_i)h_*'(x_i)}{f_*(x_i)}\right)^2}{
        \Bigl |\left(\cfrac{f_*'(x_i)}{f_*(x_i)} -
        \cfrac{g_*'(x_i)}{g_*(x_i)}\right)
        +(\mu_s-\mu) \cfrac{g_*(x_i)h_*'(x_i)}{f_*(x_i)} \Bigl |^2}
    < \sigma^{-2}. 
\end{equation}
This expression can be considered the
random interaction case equivalent of
the expression in Eq.~\eqref{homogeneous-general} 
which appears again in the denominator of the summand.
It reduces indeed to the homogenous case for $\sigma\to 0$.
\red{To be discussed.}

There are straightforward results 
that follows from this expression.
For generalized Lotka-Volterra (GLV) models,
i.e. $f(x_i)=x_i$, $g(x_i)=x_i$ and $h(x_i)=x_i$,
the expression in Eq.~\eqref{eq: random-stability}
becomes independent on the equilibria and we easily 
recover the linear stability condition
\begin{equation}
    \sigma\sqrt{N} < (\mu_s-\mu) \ .
\end{equation}
Moreover we can define an equivalence class of models
which share the same simple condition
for linear stability defined by the equivalence relation
\begin{equation}\label{eq: equiv class glv}
    \left(\cfrac{f_*'(x_i)}{f_*(x_i)} -
        \cfrac{g_*'(x_i)}{g_*(x_i)}\right)\Bigl/
        \cfrac{g_*(x_i)h_*'(x_i)}{f_*(x_i)}=c \ ,
\end{equation}
where $c$ is a constant,
resulting in the condition $\sigma\sqrt{N} < (\mu_s-\mu) +c$.

In the general case in which the dependence on the equilibria
$\mathbf{x}_*$ does not cancel out, is not possible to
obtain a clean complexity-stability relationship as before.
We can make progress in the large $N$ limit by
trasforming the sum in an integral 
$\sum_{i=1}^N\to N\int dx_*P(x_*)$, where $P(x_*)$ 
is the equilibrium distribution of the system
depending on the statistic of the interactions,
and write
\begin{equation}\label{eq: random-stability int}
    \int dx_* \cfrac{NP(x_*) \left(\cfrac{g_*(x)h_*'(x)}{f_*(x)}\right)^2}{
        \Bigl |\left(\cfrac{f_*'(x)}{f_*(x)} -
        \cfrac{g_*'(x)}{g_*(x)}\right)
        +(\mu_s-\mu) \cfrac{g_*(x)h_*'(x)}{f_*(x)} \Bigl |^2}
    < \sigma^{-2} \ .
\end{equation}
\red{There are things to clarify and discuss.}
This expression allows us to obtain insights on the stability properties of
the system in the case in which we are able to compute or estimate 
the probability distribution function $P(x_*)$.

In order to investigate the different regimes
of complexity-stability behaviour,
let us specialize to the power laws case, with 
\begin{equation}
    f(x_i)=x_i^{\alpha} \ , \quad g(x_i)=-x_i^{\beta} \ , \quad h(x_i)=x_i^{\gamma} \ .
\end{equation} 
Additional proportionality constants can be reabsorbed in the statistics
of $a$ and by a rescaling of time, moreover we chose to explicitly
factor out a minus sign in order to more straightforwardly connect
with competitive models.

We wish to consider explictly cases
for "May" and "anti-May" regimes.
An example of the former is given by GLV.
In order to explore the latter, we 
need to resort to Eq.~\eqref{eq: random-stability int} and therefore
to compute the equilibrium probability distribution function
for the system. This can be achieved in different ways,
depending among other things on specific constraint that
one can impose on the system. 
One approach is reported below, then we specialize to a choice of
the power laws exponens.

Following, e.g., Ref.~[Roy2019] we can derive from 
the evolution equation
\begin{equation}
    \dot{x_i}=x_i^{\alpha} - x_i^{\beta}\sum_{j}a_{ij}x_j^{\gamma} \, ,
\label{eq: full abg}
\end{equation}
for large $N$, a dynamical mean field theory (DMFT) 
describing the entire ensemble of $N$ variables
thorugh the evolution of the representative random variable $x$
\begin{equation}
    \dot{x} = x^{\alpha}-a_sx^{\beta+\gamma}-x^{\beta}\left( \mu N \langle x^{\gamma}\rangle + \sigma \sqrt{N} \eta\right) \, ,
\label{eq: dmft}
\end{equation}
where $a_s$ is a random variable with the statistics of $a_{ii}$,
$\langle . \rangle$ stands for expectation value
and $\eta$ is a gaussian variable
with zero mean and correlation 
$\langle x^{\gamma}(t)x^{\gamma}(s)\rangle$. 
The averages and correlations have to be computed self-consistently.
If the system admits an equilibrium (i.e. is feasible), for 
$t\to\infty$ each of the initial $N$ degrees 
of freedom will reach a final value, 
and the effective random variable describing their
also become time-independent,
$\lim_{t\to\infty}x(t)=x_*$ and we can write
\begin{equation}
    x_*^{\alpha-\beta} - a_s x_*^{\gamma}= \left( \mu N \langle x_*^{\gamma}\rangle + \sigma \sqrt{N\langle x_*^{2\gamma}\rangle}\xi\right) \, ,
\end{equation} 
where $\xi$ is a standard normal random variable (i.e. $\xi\sim\mathcal{N}(0,1)$).
The equation above can be solved for $x_*$ in if $(\alpha-\beta)/\gamma\in\mathbb{N}$,
or in general, if we set $a_s=0$, which amounts to include
all the self-interaction in the nonlinear produciton term
$x_*^{\alpha}$.
In this case the stationary solution is given by 
\begin{equation}
    x_* = \left( \mu N \langle x_*^{\gamma}\rangle + \sigma \sqrt{N\langle x_*^{2\gamma}\rangle}\xi\right)^{1/(\alpha-\beta)} \, .
\end{equation}

The equilibrium probability distribution function for $x_*$, $P(x_*)$,
can be obtained as the pushforward of the distribition of $\xi$.
The change of measure
\begin{equation}
    \left|\frac{d \xi}{d x_*}\right|=\frac{|\alpha-\beta|x_*^{\alpha-\beta-1}}{\sigma \sqrt{N\langle x_*^{2\gamma}\rangle}} \, ,
\end{equation}
leads to
\begin{equation}\label{eq: dist general}
    P(x_*)=\frac{|\alpha-\beta|x_*^{\alpha-\beta-1}}{\sqrt{2\pi\sigma^2 N\langle x_*^{2\gamma}\rangle}}
    \exp{\left\{-\frac{(x_*^{\alpha-\beta}-\mu N\langle x_*^{\gamma}\rangle)^2}{2\sigma^2N\langle x_*^{2\gamma}\rangle}\right\}} \, ,
\end{equation}
where the expectations have to be computed self-consistently, 
taking also in account normalization constraint.

Let us focus to the case $\alpha=1$, $\beta=2$ and $\gamma=1$, with
$a_s=0$ in order to leverage the equilibrium solution in Eq.~\eqref{eq: dist general}.

\dots

\subsection{Simulations}

\begin{itemize}
    \item Equilibrium distribution and histogram for "anti-May" example
    \item ($\mu N$, $\sigma\sqrt{N}$) plane for "May" and "anti-May" examples
        with stability lines.
    \item ($\alpha$, $\beta$) plane with homogeneous interactions line.
    \item Others?
\end{itemize}

\section{Conclusions}


\end{document}
%
% ****** End of file apssamp.tex ******

