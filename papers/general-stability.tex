% ****** Start of file apssamp.tex ******
%
%   This file is part of the APS files in the REVTeX 4.2 distribution.
%   Version 4.2a of REVTeX, December 2014
%
%   Copyright (c) 2014 The American Physical Society.
%
%   See the REVTeX 4 README file for restrictions and more information.
%
% TeX'ing this file requires that you have AMS-LaTeX 2.0 installed
% as well as the rest of the prerequisites for REVTeX 4.2
%
% See the REVTeX 4 README file
% It also requires running BibTeX. The commands are as follows:
%
%  1)  latex apssamp.tex
%  2)  bibtex apssamp
%  3)  latex apssamp.tex
%  4)  latex apssamp.tex
%
\documentclass[%
 reprint,
%superscriptaddress,
%groupedaddress,
%unsortedaddress,
%runinaddress,
%frontmatterverbose, 
%preprint,
%preprintnumbers,
%nofootinbib,
%nobibnotes,
%bibnotes,
 amsmath,amssymb,
 aps,
%pra,
%prb,
%rmp,
%prstab,
%prstper,
%floatfix,
]{revtex4-2}

\usepackage{graphicx}% Include figure files
\usepackage{dcolumn}% Align table columns on decimal point
\usepackage{bm}% bold math
%\usepackage{hyperref}% add hypertext capabilities
%\usepackage[mathlines]{lineno}% Enable numbering of text and display math
%\linenumbers\relax % Commence numbering lines

%\usepackage[showframe,%Uncomment any one of the following lines to test 
%%scale=0.7, marginratio={1:1, 2:3}, ignoreall,% default settings
%%text={7in,10in},centering,
%%margin=1.5in,
%%total={6.5in,8.75in}, top=1.2in, left=0.9in, includefoot,
%%height=10in,a5paper,hmargin={3cm,0.8in},
%]{geometry}

\begin{document}


\title{The two faces of the complexity-stability relationship}

\author{Ann Author}
 \affiliation{MPI MiS}%Lines break automatically or can be forced with \\


\date{\today}% It is always \today, today,
             %  but any date may be explicitly specified

\begin{abstract}
Robert May famously predicted that the likelihood of an interacting system to be stable decreases with its complexity. We show that the opposite relationship is equally possible.   
\end{abstract}


\maketitle

In 1970 Robert May used a simple random matrix argument to predict that large, complex ecosystems are not generically stable. While this conclusion is not supported by empirical observation, May's prediction that "complexity begets instability" quickly became a staple of complex systems theory. OVERVIEW OF LITERATURE

May's argument goes as follows. For a dynamical system to be linearly stable, the eigenvalues of its Jacobian matrix evaluated at equilibrium must be have negative real part. Let us assume that the Jacobian can be written as $A = B - I$, where $A$ represent random interactions and $-I$  

May's complexity-stability relationship---the prediction that large complex systems are unlikely to be stable, or ``complexity begets instability"---is a landmark of complex systems theory. 

$A = N^\alpha B - N^\beta I$

%\tableofcontents
Consider the dynamics of $N$ degrees  
\begin{equation}
  \dot{x}_i = r_ix_i[s_i(x_i) + c_i(x_{-i})]
\end{equation}
\begin{equation}
  J_{ij}(x) = r_ix_i[\delta_{ij}  s_i'(x_i) + (1-\delta_{ij})\partial_j c_i(x_{-i})]
\end{equation}


\end{document}
%
% ****** End of file apssamp.tex ******