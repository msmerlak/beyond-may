\emph{Homogeneous interactions.---}Under what condition does~\eqref{dynamics} admit a (linearly) stable equilibrium? 
We begin with the simple case where all self-interactions have the same strength $A_{ii} = \mu_s$, and similarly for cross-interactions $A_{ij} = \mu$ ($i\neq j$).
Defining 
\begin{equation}
    \psi(x) \equiv (\mu_s - \mu) -  \left(\frac{f'(x)}{f(x)} - \frac{g'(x)}{g(x)}\right)\frac{f(x)}{g(x)h'(x)},
    \label{eq: psi}
\end{equation}
we find that stability of the homogeneous equilibrium $x_i^* = x^*$ requires $\psi(x^*)g(x^*)h'(x^*) > 0$. This reduces to $\psi(x^*)>0$ if we assume (and we will, for simplicity, in the following) $g$ and $h$ are positive, increasing functions. This condition involves the relative strength of diagonal and off-diagonal interactions ($\mu_s - \mu$), but also on the relative growth rate of the production and response functions near the equilibrium ($f'(x^*)/f(x^*) - g'(x^*)/g(x^*)$).

With power laws, the condition $\psi(x^*)>0$ evaluates to 
\begin{equation}
    (\alpha - \beta)(N-1) < \gamma(\mu_s/\mu- 1) - (\alpha - \beta)(\mu_s/\mu),
\end{equation}
leading to three different regimes:
\begin{itemize}
    \item If $\alpha = \beta$, stability requires $\mu_s > \mu$, i.e.
    self-interactions must be stronger than cross-interactions.
    This is the usual conclusion drawn from the Lotka-Volterra model.
    \item If $\alpha > \beta$, stability places an upper bound on $N$: the more complex the system, the less likely to be stable.
    We can call this ``May" behavior.
    \item If $\alpha < \beta$, stability places an lower bound on $N$: the more complex the system, the more likely to be stable.
    This is ``anti-May" behavior.
\end{itemize}
