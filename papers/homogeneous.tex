
Under what condition does \eqref{dynamics} admit a linearly stable equilibrium? 
We begin by considering the simple case where all self-interactions have the same strength $a_{ii} = a_{\textrm{s}}$, and similarly for cross-interactions $a_{ij} = a_{\textrm{c}}$ ($i\neq j$).
Under these assumptions, a straightforward computation shows that a homogeneous equilibrium $\mathbf x_*$ will be linearly stable if  
\begin{equation}\label{homogeneous-general}
   \frac{f'(x^*)}{f(x^*)} - \frac{g'(x^*)}{g(x^*)} + (a_{\textrm{s}} - a_{\textrm{c}})\,\frac{g(x^*)h'(x^*)}{f(x^*)} < 0.
\end{equation}
Thus, stability of an equilibrium depends on the relative strength of self- and cross-interactions ($a_{\textrm{s}} - a_{\textrm{c}}$), but also on the relative growth rate of the production and response functions near that equilibrium ($f'_*/f_* - g'_*/g_*$).
With power laws, \eqref{homogeneous-general} evaluates to 
\begin{equation}
    (\alpha - \beta)(N-1) < \gamma(a_{\textrm{s}}/a_{\textrm{c}}- 1) - (\alpha - \beta)(a_{\textrm{s}}/a_{\textrm{c}}),
\end{equation}
leading to three different regimes:
\begin{itemize}
    \item If $\alpha = \beta$, stability requires $a_{\textrm{s}} > a_{\textrm{c}}$, i.e.
    self-interactions must be stronger than cross-interactions.
    This is the usual conclusion drawn from the Lotka-Volterra model.
    \item If $\alpha > \beta$, stability places an upper bound on $N$: the more complex the system, the less likely to be stable.
    We can call this "May" behavior.
    \item If $\alpha < \beta$, stability places an lower bound on $N$: the more complex the system, the more likely to be stable.
    This is "anti-May" behavior.
\end{itemize}

