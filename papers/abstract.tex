Robert May famously used random matrix theory to predict that large, complex systems cannot admit stable fixed points. 
However, this seemingly general conclusion is not supported by empirical observation: from cells to biomes, biological systems are large, complex---and, by and large, stable.
In this paper, we revisit May's argument in the light of recent developments in both ecology and random matrix theory. 
Using a non-linear generalization of the Lotka-Volterra model, we show that there are in fact two kinds of complexity-stability relationships in disordered dynamical systems: 
if self-interactions grow faster with density than cross-interactions, complexity is destabilizing; but if cross-interactions grow faster than self-interactions, complexity is stabilizing.
Our result shows that May's principle that "complexity begets instability" is not a general property of complex systems; instead, it is a property of a class of weakly cross-regulated complex systems.