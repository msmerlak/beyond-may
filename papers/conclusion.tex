
The relationship between complexity and stability in high-dimensional dynamical systems has been a longstanding puzzle, in ecology and in other fields. 
In this letter, we have showed that the condition $\sigma\sqrt{N}< \mu_s - \mu$ (popularized by May's slogan ``complexity begets instability") does not provide a complete picture of the relationship between complexity and stability. 
In particular, we have seen that generalized Lotka-Volterra model, often cited in support of May's prediction, corresponds to a special cancellation in the more general stability condition $N\langle (\sigma/\psi)^2\rangle < 1$.
In models where the self-dynamics is more strongly regulated than the cross-interactions ($\alpha < \beta$, which might be more appropriate for ecological modeling \cite{Hatton2023}), the opposite behavior is observed: stability becomes \emph{more} likely with increasing diversity $N$. 