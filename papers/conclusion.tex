\emph{Discussion.---}The relationship between complexity and stability in high-dimensional dynamical systems has been a longstanding puzzle, in ecology and in other fields. 
In this letter, we have showed that the condition $\sigma\sqrt{N}< \mu_s - \mu$ (popularized by May's slogan ``complexity begets instability") does not provide a complete picture of the relationship between complexity and stability. 
In particular, we have seen that generalized Lotka-Volterra model, often cited in support of May's prediction, corresponds to a special cancellation in the more general stability condition $N\langle (\sigma/\psi)^2\rangle < 1$.
In models where the self-dynamics is more strongly regulated than the cross-interactions, the opposite behavior is observed: stability becomes \emph{more} likely with increasing diversity $N$.

GLV is special also for another reason: for homogenous interaction, dependence on the number of degrees of freedom is lost in the stability condition.
We showed that in general this dependence is present, and its effect on the complexity-stablity property is preserved for heterogeneous interaction. 
The simple condition $\psi(x^*)g(x^*)h'(x^*)>0$ can therefore be employed to gain informations on systems desceribed by Eq.~\eqref{dynamics}.
Future work can be done in the direction of extending the equations for testing complexity-stability for cases in which we can not decompose the two-body interactions, or to include higher order interaction, generalizing, e.g., the analysis of recent results~\cite{Gibbs2022} based on GLV. 


Our general conclusion that for large random dynamical systems, an increase in dimensionality can lead to either stabilization or destabilization, without invoking specific system topology, bares significant implications for the study and control of complex systems.